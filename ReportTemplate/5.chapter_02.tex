\chapter{Images Component}

\section{Video: Displaying Images in React Native}
\subsection{Images Component}
\begin{itemize}
    \item It is a core component used to displaying different types of images.
    \item Display:
    \begin{itemize}
        \item Static images from Resources
        \item Temporary local images
        \item Images from local disk
        \item Network images
    \end{itemize}
\end{itemize}

\subsection{Using}
\begin{itemize}
    \item Create a img folder within the project folder.
    \item Syntax:
    \begin{lstlisting}[language=Java, numbers=none]
        <View>
            <Image
                style={}
                source={require('url')}
            />
        </View>
    \end{lstlisting}
    \item New props to pass in the style: \texttt{resizeMode}, which determines how to resize the image when the frame doesn't match the image's dimensions.
    \item Option: 'strech', 'repeat', 'cover', 'center'
    \item If the images is hosted on network:
    \begin{lstlisting}[language=Java, numbers=none]
        <View>
            <Image
                style={}
                source={{
                    uri: 'url'
                }}
            />
        </View>
    \end{lstlisting}
    \item React Native also provides \texttt{loadingIndicatorSource}, which represents resources used to render the loading indicator, this is displayed until the resources is fully loaded.
    \item \textbf{Practice Question: } You would like to add some images to your mobile app using the 
    \texttt{Image} component. Which sources can you retrieve images from? Choose all that apply. \\
    $\rightarrow$ Answer:
    \begin{itemize}
        \item A local drive on your computer
        \item A web URL
        \item Temporary Images
        \item Resources
    \end{itemize}
\end{itemize}

\subsection{Support images type}
\begin{itemize}
    \item png
    \item jpg
    \item gif
    \item webp
    \item These type are supported for both iOS and Android.
    \item In addition, iOS supports for psd, raw and some other uncompressed image format.
\end{itemize}

\section{Video: Using Image Component}
\begin{itemize}
    \item \textbf{Practice Question: } In your project folder, you have a folder called \texttt{img} inside of which you have uploaded an image file titled \texttt{header.png}. How would you render this image using an \texttt{Image} component? Choose the appropriate line of code from the following.
    $\rightarrow$
    \begin{lstlisting}[language=Java, numbers=none]
        <Image source=/{require('img/header.png')/} />
    \end{lstlisting}
\end{itemize}

\begin{figure}[H]
    \centering
    \includegraphics[width=0.2\textwidth]{images/images1.png}
    \caption{Using Image Component}
\end{figure}

\section{Video: Styling an Image within the app}
\begin{itemize}
    \item This part is to use the \texttt{resizeMode} method that I have already used in the above part.
    \item \textbf{Practice Question:} Which one of the resize modes is described in the following statement:  It will scale the image uniformly so that both the dimensions – the width and the height of the image will be equal to or less than the corresponding dimension of the view. 
    $\rightarrow$ Contain
\end{itemize}

\section{Video: Passing props to the Image Component}
\begin{itemize}
    \item Using \texttt{accessible} props, a Boolean.
    \item When the Boolean is true, it indicates that the image is an accessibility element.
    \item \textbf{Practice Question:} If you want an image to fill the entire box, which of the following resize mode props will you use?  
    $\rightarrow$ Cover
\end{itemize}

\section{Reading: Exploring props to the Image Component}
\begin{figure}[H]
    \centering
    \begin{minipage}[b]{0.32\textwidth}
        \centering
        \includegraphics[width=0.5\textwidth]{images/contain.png}
        \caption{resizeMode contain}
    \end{minipage}
    \hfill
    \begin{minipage}[b]{0.32\textwidth}
        \centering
        \includegraphics[width=0.5\textwidth]{images/cover.png}
        \caption{resizeMode cover}
    \end{minipage}
    \hfill
    \begin{minipage}[b]{0.32\textwidth}
        \centering
        \includegraphics[width=0.5\textwidth]{images/stretch.png}
        \caption{resizeMode strech}
    \end{minipage}
\end{figure}

\begin{figure}[H]
    \centering
    \begin{minipage}[b]{0.32\textwidth}
        \centering
        \includegraphics[width=0.4\textwidth]{images/center.png}
        \caption{resizeMode center}
    \end{minipage}
    \hfill
    \begin{minipage}[b]{0.32\textwidth}
        \centering
        \includegraphics[width=0.4\textwidth]{images/repeat.png}
        \caption{resizeMode repeat}
    \end{minipage}
\end{figure}

\section{Video: Setting Background Images}
\begin{itemize}
    \item imageBackground component, which can utilize display background images.
    \item This component support for both iOS and Android.
    \item Its inherit all props from the Images.
    \item \textbf{Practice Question:} When an image is rendered as a background image, it implies that you can have text and other content on top of it. True or false?
    \item $\rightarrow$ True
\end{itemize}

\section{Reading: Exercise: Displaying Images in your app}
\begin{figure}[H]
    \centering
    \includegraphics[width=0.2\textwidth]{images/ex2.png}
    \caption{Exercise: Displaying Images in your app}
\end{figure}

\section{Practice Assignment: Self review: Displaying Images in your app}
\begin{figure}[H]
    \centering
    \includegraphics[width=0.5\textwidth]{images/sr-2.png}
    \caption{Self review: Displaying Images in your app}
\end{figure}

\section{Practice Assignment: Knowledge Check: Images in React Native}
\begin{figure}[H]
    \centering
    \includegraphics[width=0.5\textwidth]{images/kc-2.png}
    \caption{Knowledge Check: Images in React Native}
\end{figure}