\chapter{Blur the image}

\section{Idea}
\begin{itemize}
    \item Blur the image using Box blur.
    \begin{itemize}[label=$\circ$]
        \item Box blur \cite{box_blur} blurs an image by replacing each pixel with the average value of the pixels surrounding it in a square region (kernel - k x k matrix). 
        \item The size of the kernel is determined by the radius r.
        \item Kernel matrix follows the formula:
        \[
        \text{kernel} = \frac{1}{(2 \cdot r + 1)^2}
        \begin{bmatrix}
        1 & 1 & \cdots & 1 \\
        1 & 1 & \cdots & 1 \\
        \vdots & \vdots & \ddots & \vdots \\
        1 & 1 & \cdots & 1 \\
        \end{bmatrix}_{(2r + 1) \times (2r + 1)}
        \]
        \item For example: with r = 2, kernel is:
        \[
        \frac{1}{25} \begin{bmatrix}
            1 & 1 & 1 & 1 & 1 \\
            1 & 1 & 1 & 1 & 1 \\
            1 & 1 & 1 & 1 & 1 \\
            1 & 1 & 1 & 1 & 1 \\
            1 & 1 & 1 & 1 & 1
        \end{bmatrix}
        \]
    \end{itemize}
\end{itemize}

\section{Processing}
\begin{enumerate}
    \item Convert image to \texttt{float32} to avoid overflow during the process.
    \item Check for removing the Alpha channel, keep only 3 RGB channels.
    \item Input a 2D image and the radius of kernel (optional, r = 2 by default).
    \item Initialize a \( (2 \cdot r + 1) \times (2 \cdot r + 1) \) kernel matrix following the above formula.
    \item Convert the image to grayscale to reduce the necessary calculation during the blur process.
    \item Pad add pixels at the edge to ensure that when scanning the kernel to the edge, there are still enough values to calculate.
    \item Traversing for each pixel and calculating the average value of the pixels surrounding it in kernel.
    \item To ensure valid pixel values, the result is clipped to the range \([0, 255]\). Then convert to \texttt{uit8} for correct display.
\end{enumerate}

\section{Result}
\begin{itemize}
    \item Original image/Blur with r = 2/Blur with r = 3
    \begin{center}
        \includegraphics[width=0.3\textwidth]{images/img.png}
        \includegraphics[width=0.3\textwidth]{images/img_blur.png}
        \includegraphics[width=0.3\textwidth]{images/img_blur_2.png}
    \end{center} 
    \begin{center}
        \includegraphics[width=0.3\textwidth]{images/img2.jpg}
        \includegraphics[width=0.3\textwidth]{images/img2_blur_2.jpg}
        \includegraphics[width=0.3\textwidth]{images/img2_blur.jpg}
    \end{center} 
\end{itemize}