\chapter{Sharpen the image}

\section{Idea}
\begin{itemize}
    \item Using sharpen kernel to highlights the edges in an image.
    \item Sharpen kernel \cite{box_blur}:
    \[
        \begin{bmatrix}
            0 & -1 & 0 \\
            -1 & 5 & -1 \\
            0 & -1 & 0
        \end{bmatrix}
    \]
    \item Purpose:
    \begin{itemize}
        \item Increase center point weight (value 5).
        \item Decreases the neighborhood weight (value -1).
        \begin{flushleft}
            $\Rightarrow$ \text{Highlight details, sharper edges.}
        \end{flushleft}
    \end{itemize}
\end{itemize}

\section{Processing}
\begin{enumerate}
    \item Convert image to \texttt{float32} to avoid overflow during the process.
    \item Check for removing the Alpha channel, keep only 3 RGB channels.
    \item Define sharpen kernel following the above matrix.
    \item Convert the image to grayscale to reduce the necessary calculation during the blur process.
    \item Pad add pixels at the edge to ensure that when scanning the kernel to the edge, there are still enough values to calculate.
    \item Traversing for each pixel and applying the kernel.
    \item To ensure valid pixel values, the result is clipped to the range \([0, 255]\). Then convert to \texttt{uit8} for correct display.
\end{enumerate}

\section{Result}
\begin{itemize}
    \item Original image/Image after being sharpened
    \begin{center}
        \includegraphics[width=0.3\textwidth]{images/img_gray.png}
        \includegraphics[width=0.3\textwidth]{images/img_sharpen.png}
    \end{center} 
    \begin{center}
        \includegraphics[width=0.3\textwidth]{images/img2_gray.jpg}
        \includegraphics[width=0.3\textwidth]{images/img2_sharpen.jpg}
    \end{center} 
    \begin{center}
        \includegraphics[width=0.3\textwidth]{images/img3_gray.jpg}
        \includegraphics[width=0.3\textwidth]{images/img3_sharpen.jpg}
    \end{center} 
\end{itemize}