\chapter{Hooks}

\section{Video: What are Hooks?}
\begin{itemize}
    \item useState hook is used to manage the state within a component and to keep track of this.
    \item \textbf{Practice Question:} Observe the following line of code which sets up the useState hook:  
    
    const [showMenu, setShowMenu] = useState(false);
    
    Which element in this code represents the state variable? 
    $\rightarrow$ showMenu
    \item In addition to the hooks that are built-in to React, you can also build your own hooks which will let you extract custom component logic into reusable functions
    \item The biggest benefit of hooks overall is the readability and simplicity that they provide for your code.
\end{itemize}

\section{Video: Using the useColorScheme hook}
\begin{itemize}
    \item Provides and subscribes to color scheme updates from the appearance module in React Native.
    \item Pracical use: Apply styling changes in the app based on whether the user's device is set to light theme or dark theme.
    \item Value: light, dark, null
    \textbf{Practice Question:} You want to apply the \texttt{useColorScheme} hook to adjust your app’s color scheme based on the device's color theme setting. Which color theme options are available to you? Select all that apply.
    $\rightarrow$ null, light, dark
\end{itemize}

\begin{figure}[H]
    \centering
    \includegraphics[width=0.2\textwidth]{images/light-theme.png}
    \caption{UI with light theme}
\end{figure}

\section{Reading: Exploring the useColorScheme hook}
This part is the same progress with above part.

\section{Using useWindowDimensions hook}
\begin{itemize}
    \item This hook provides information about the window size.
    \item From this information, we can determine how elements could appear on Windows of different sizes and make mor einformed decisions on how to adjust size scale.
    \item Using:
    \begin{lstlisting}[language=Java, numbers=none]
        const {width, height, fontScale} = useWindowDimensions();
    \end{lstlisting}
    \textbf{Practice Question:} You're creating an app using React Native, and you need to make some design decisions. For what purpose would you use the \texttt{useWindowDimensions} hook? Check all that apply.
    $\rightarrow$ to determine the window height, width, font scale.
\end{itemize}

\section{Reading: Explore the useWindowDimensions hook}
\begin{figure}[H]
    \centering
    \includegraphics[width=0.2\textwidth]{images/windowDimension.png}
    \caption{The window dimension of my virtual phone}
\end{figure}

\section{Video: Using other community hooks}
\begin{itemize}
    \item React Native is an open source, meaning that the developer community can contribute new hooks and extend React Native.
    \item To access, first we need to access the web page at Github.com/react-native-community/hooks.
    \item Installation:
    \begin{lstlisting}[numbers=none]
        npm install @react-native-community/hooks
    \end{lstlisting}
    \textbf{Practice Question:} You have a React Native app that utilizes location tracking, but you would like this feature to be disabled when a user pushes the app to the background. Which community hook could you use to set up this functionality? 
    $\rightarrow$ useAppState
\end{itemize}

\section{Reading: Exploring other community hooks}
\subsection{useDeviceOrientation Hook}
\begin{itemize}
    \item This hook can determine if the user’s mobile device is viewed in landscape or portrait mode. 
    \item You can utilize this hook to determine the orientation if your app needs to support both modes.
    \begin{lstlisting}[language=Java, numbers=none]
        import { useDeviceOrientation } from '@react-native-community/hooks' 

        const orientation = useDeviceOrientation() 
        
        console.log(orientation.portrait) // Boolean
        console.log(orientation.landscape) // Boolean
    \end{lstlisting}
\end{itemize}

\subsection{useAppState Hook}
\begin{itemize}
    \item This hook is used to determine the current app state. 
    \item It can be active, background or inactive (iOS only). 
\end{itemize}

\subsection{useClipboard Hook}
\begin{itemize}
    \item Store text from the clipboard within the app.
    \item Using:
    \begin{lstlisting}[language=Java, numbers=none]
        import { useClipboard } from '@react-native-community/hooks' 

        const [data, setString] = useClipboard() 

        <Text>{data}</Text> 

        <Button title='Update Clipboard' onPress={() => setString('new clipboard data')}>Set Clipboard</Button> 
    \end{lstlisting}
\end{itemize}

\section{Reading: Exercise: Hooks in React Native}
This part is to implement the light/dark mode that I have done in the section 3.2

\section{Practice Assignment: Self review: Hooks in React Native}
\begin{figure}[H]
    \centering
    \includegraphics[width=0.5\textwidth]{images/sr-3.png}
    \caption{Self review: Hooks in React Native}
\end{figure}

\section{Practice Assignment: Knowledge Check: Hooks in React Native}
\begin{figure}[H]
    \centering
    \includegraphics[width=0.5\textwidth]{images/kc-3.png}
    \caption{Knowledge Check: Hooks in React Native}
\end{figure}

\section{Module summary: Pressable, Images and Hooks in React Native}
\begin{itemize}
    \item Pressable components
    \begin{itemize}
        \item Explain Pressable component
        \item Identify Pressable methods
        \item Explain how to represent a Pressable component
        \item Create Pressable that responds to press
        \item Track Pressable state
    \end{itemize}
    \item Images components
    \begin{itemize}
        \item Explain Image component
        \item Identify images sources
        \item Identify supported Image types
        \item Create Image component
        \item Style an Image
    \end{itemize}
    \item Hooks in React Native
    \begin{itemize}
        \item Describe uses for hooks
        \item Describe hook structure
        \item useColorScheme hook
        \item useWindowDimensions hook
        \item Community hooks
    \end{itemize}
\end{itemize}

\section{Practice Assignment: Module quiz: Pressable, Images and Hooks in React Native}
\begin{figure}[H]
    \centering
    \includegraphics[width=0.5\textwidth]{images/module-quiz.png}
    \caption{Module quiz: Pressable, Images and Hooks in React Native}
\end{figure}