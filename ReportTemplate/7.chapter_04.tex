\chapter{Views, Text and Scrollable components}

\section{Video: What are View and Text components?}
\subsection{View}

\begin{itemize}
    \item View is the basic building block of the user interface.
    \item It is a small rectangular element on the screen, houses all elements within it.
    \item The view component is like a non-scrolling div tag in the web world.
\end{itemize}

\subsection{Text}
\begin{itemize}
    \item A basic core component from React Native that is used for displaying text.
    \item Support styling, nesting and touch handling.

     \item \textbf{Practive Question:} The \texttt{View} component can be nested inside other views and can have zero or many children within it. True or false?
    $\rightarrow$ True
\end{itemize}

\section{Video: Using the View and Text Components}
\begin{itemize}
    \item Using \text{flex: 0.5} $\rightarrow$ the child component take 50\% of parent's space.
    \item The text must be rendered within a text components.
    \item Use nested text components, the child text components can be inherited the style from parent's components, we can also override the style.
    \item Using \texttt{numberOfline={1}} to avoid wrap.
    \item \textbf{Practive Question:} You have created a View component for your app and would like it to take 50\% of the space on the screen. Which of the following styling parameter settings would you apply to achieve this?
    $\rightarrow$ flex: 0.5
\end{itemize}

\section{Reading: Exercise: Build a React Native Screen}
\begin{figure}[H]
    \centering
    \includegraphics[width=0.2\linewidth]{images/ex2.pdf}
    \caption{Exercise: Build a React Native Screen}
    \label{fig:placeholder}
\end{figure}

\section{Self Review: Build a React Native screen}
\begin{figure}[H]
    \centering
    \includegraphics[width=0.5\linewidth]{images/self-review-2.pdf}
    \caption{Self Review: Build a React Native Screen}
    \label{fig:placeholder}
\end{figure}

\section{Video: What is the ScrollView component?}
\subsection{Problem}
\begin{itemize}
    \item The menu items do not fit within the small mobile screen.
    \item Users should be able to scroll up and down the menu.
\end{itemize}

\subsection{ScrollView}
\begin{itemize}
    \item The ScrollView component has only one rule. It must be bounded by a height in order to work.
    \item The parent node what wrap the ScrollView have to set a fix height.
    \item Use this core component, we have both scroll view for iOS and Android.

    \item \textbf{Practive Question:} Which of the following are true statements about the ScrollView component? Check all that apply.
    $\rightarrow$ ScrollView is a core React Native component and comes as a part of the React Native Package, It must be bounded by a height in order to work.
\end{itemize}

\section{Video: Using the ScrollView component}
\begin{figure}[H]
    \centering
    \includegraphics[width=0.2\linewidth]{images/scroll-view.pdf}
    \caption{Scroll View practice}
    \label{fig:placeholder}
\end{figure}

\section{Reading: Exercise: Build a scrollable component}
\begin{figure}[H]
    \centering
    \includegraphics[width=0.2\linewidth]{images/scroll-view-practice.pdf}
    \caption{Exercise: Build a scrollable component}
    \label{fig:placeholder}
\end{figure}

\section{Practice Assignment: Self review: Build a scrollable component}
\begin{figure}[H]
    \centering
    \includegraphics[width=0.5\linewidth]{images/self-review-3.pdf}
    \caption{Self review: Build a scrollable component}
    \label{fig:placeholder}
\end{figure}

\section{Practive Assignment: Knowledge check: Views, Text and Scrollable components}
\begin{figure}[H]
    \centering
    \includegraphics[width=0.5\linewidth]{images/knowledge-checking-3.pdf}
    \caption{Knowledge check: Views, Text and Scrollable components}
    \label{fig:placeholder}
\end{figure}