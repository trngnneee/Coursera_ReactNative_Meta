\chapter{React Navigation}

\section{Video: What is React Navigation}
\subsection{Problem}
\begin{itemize}
    \item Moving between screens is referred to as navigation or routing in mobile apps.
    \item Mobile apps have multiple screens and the user's experience relies on smooth navigation between these screens.
    \item React Native does not have inbuilt navigation, it must pick a navigation library to implement moving between the screens and the app.
    $\rightarrow$ React Navigation comes onto the screen
\end{itemize}

\subsection{React Native Navigation}
\begin{itemize}
    \item React Navigation is the most popular navigation library for React Native apps today.
    \item It is easy to use, customizable based on your needs, and has platform-specific components for both iOS and Android.
    \item Can be used for both React Native CLI apps and React Native built with Expo.
    \item It provides a Native Stack Navigator that can transition between screens and manage the navigation history.
\end{itemize}

\subsection{Benefit of using React Native Navigation}
\begin{itemize}
    \item The difference between routing and navigation works in web browsers versus React Navigation for mobile app: \textbf{React Navigation provides the gestures and animations.}
    \item React Native provides Screens, which provide native navigation container components
    \item Its also provides gesture handler, which make React Native more compatible with native touch and gesture system.
    \item Which brings:
    \begin{itemize} 
        \item Easier to push OTA updates.
        \item Easier to debug.
        \item Possible to customize navigator components.
    \end{itemize}
\end{itemize}

\subsection{Trade off}
\begin{itemize}
    \item Some of the navigators don't directly use the native navigation APIs on iOS and Android. 
    \item While gaining access to new features after upgrading is generally good news, it's possible that code using older APIs will no longer work as expected.
\end{itemize}

\subsection{Other commonly used library}
\begin{itemize}
    \item React-native-router-flux
    \item React-native-navigation
\end{itemize}

\textbf{Practice Question:} React Navigation is the most popular navigation library for React Native apps. Which of the following are characteristics of this library? Select all that apply.
$\rightarrow$ Answer:
\begin{itemize}
    \item iOS-specific components.
    \item Customizable navigators 
    \item Android-specific components.
\end{itemize}

\section{Reading: Installation and Setup of React Navigation}
\begin{lstlisting}[language=Java, numbers=none]
    npm install @react-navigation/native

    npx expo install react-native-screens react-native-safe-area-context
\end{lstlisting}

\section{Video: Setting up React Navigation}
\textbf{Documentation:} https://reactnavigation.org/
\textbf{Practice Question:} What are the base packages you would need to install to setup React Navigation within your project? Select all that apply.
$\rightarrow$ Answer:
\begin{itemize}
    \item @react-navigation/native
    \item react-native-safe-area-context
    \item react-native-screens
\end{itemize}

\section{Video: Using the Stack.Navigator}
\textbf{Practice Question:} The name given to a screen within a Stack.Navigator must match the name of the component that it calls. True or false?  
$\rightarrow$ Answer: False
\begin{figure}[H]
    \centering
    \includegraphics[width=0.2\textwidth]{images/ex1.png}
    \caption{Using the Stack.Navigator}
\end{figure}

\section{Reading: Creating and Configuring Native Stack Navigator}
This part is the same progress of using Stack.Navigator that I have done in the above section.

\section{Video: Approaches to Passing Props to Screen}
\begin{itemize}
    \item To change the title of the Stack Screen, use in Stack.Screen:
    \begin{lstlisting}[language=Java, numbers=none]
        options={{ title: '...' }}
    \end{lstlisting}

    \item To set the root Stack that would be viewed when user open the first time, use in Stack.Navigator:
    \begin{lstlisting}[language=Java, numbers=none]
        initialRouteName='...'
    \end{lstlisting}
\end{itemize}
\begin{figure}[H]
    \centering
    \includegraphics[width=0.2\textwidth]{images/ex2.png}
    \caption{Passing prop into Stack}
\end{figure}

\section{Reading: Passing props to Screen}
This part is the same progress that I have done in the above section.

\section{Reading: Exercise: Set up Routes}
\begin{figure}[H]
    \centering
    \includegraphics[width=0.2\textwidth]{images/ex3.png}
    \caption{Exercise: Set up Routes}
\end{figure}

\section{Practice Assignment: Self review: Set up Routes}
\begin{figure}[H]
    \centering
    \includegraphics[width=0.5\textwidth]{images/sr1.png}
    \caption{Self review: Set up Routes}
\end{figure}

\section{Practice Assignment: Knowledge Check: Introduction to React Navigation}
\begin{figure}[H]
    \centering
    \includegraphics[width=0.5\textwidth]{images/kc1.png}
    \caption{Knowledge Check: Introduction to React Navigation}
\end{figure}